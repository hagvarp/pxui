\newenvironment{abstract}%
{\cleardoublepage\addcontentsline{toc}{chapter}{Abstract}\null\vfill\begin{center}%
\bfseries\abstractname\end{center}}%
{\vfill\null}
\begin{abstract}
Kunningartøkni, programmering og tól/dimsar er blivi lættari og lættari atkomuligt fyri forbrúkaran, also øll eiga onkran lítlan dims. Við alnótuni og microteldum, sum Arduino, kunnu øll gera teirra egnu dimsar.  

Hetta projektið snýr seg um at nýta ein Arduino Uno og fleiri smærri komponentar, til at gera ein lítlan einklan radara. Hetta fyri at skilja tey grundleggjandi tingini ísmb. við programmering og at nýta tað til at gera okkurt funktionelt ting, td. ein dims. 
\end{abstract}

\cleardoublepage\addcontentsline{toc}{chapter}{Contents}
